\documentclass{article}

% Here I'm putting comments in the preamble because the environment name comes up at the beginning of the body below.
% A "document" is an environment.  It's what we've been using so far, and it appears to be necessary.
% Let's explore some more.

\usepackage{amsmath} % let's see if I actually have this package.

\begin{document}

Environments!

\begin{center}
This is the ``center" environment.  It centers text on the page.
\end{center}

\begin{flushright}
This is the ``flush right" environment.
\end{flushright}

\begin{flushleft}
This is the ``flush left" environment.
\end{flushleft}

\begin{large}
This is the `large' environment.
\end{large}

\begin{Large}
This is the `Large' environment.
\end{Large}

\begin{small}
This is the `small' environment.
\end{small}

---------------

The `equation' environment.  First, three equivalent notations for displayed equations.

$$x^2 + y^2 = 1$$

\[ x^2 + y^2 = 1\]

\begin{equation}
x^2 + y^2 = 1
\end{equation}

Note that the `equation' environment automatically numbers equations!  I can give an equation an alias, or a `label' which is like defining a variable name that refers to that equation globally.  For example,

\begin{equation}\label{trigID}
\sin^2 x + \cos^2 x = 1
\end{equation}

Now I can refer to equation \ref{trigID} solely by its label so I don't have to worry about updating the equation numbers.  This is done automatically!

Now let's check out the `equation*' environment from the `amsmath' package.

\begin{equation*}\tag{Eq 4}\label{eq4}
x^5 + y^5 = 1
\end{equation*}
Now references to \ref{eq4} show up using the tag I specified.

---------------

The `align' environment.  First, consider this example:

\[ 3x + 2y - z = 10\]
\[ 2x + y -5z = 8 \]
\[ -x + 5y + 9z = 0 \]

Note their alignment.  It's all wonky, right?  We normally want to align equations by their equality sign, as in:

% The `align' environment comes from the `amsmath' package in the preamble.

\begin{align}
3x + 2y - z &= 10 \\ % note the presence of the '&' which indicates the following character is to be used for alignment.  The '\\' indicates there is a next line.
2x + y -5z &= 8 \\
-x + 5y + 9z &= 0 
\end{align}
Also note that the `align' environment also numbers equations like the `equation' environment.  The `align*' environment does not number by default (similar to the `equation*' environment).

Let's practice with the Triangle Proof:

\begin{align*}
|x + y|^2 &= (x + y)^2 \\
 &= x^2 + 2xy + y^2 \\
 &= |x|^2 + 2xy + |y|^2 \\
 &\leq |x|^2 + 2|xy| + |y|^2 \\
&= |x|^2 + 2|x| |y| + |y|^2 \\
 &= (|x| + |y|)^2
\end{align*}

Another example, integration by parts:

$$\int x \cos x dx$$

\begin{align*}
u &= x & v &= \sin x dx \\ % the first and third '&' are to align by '=' signs.  The second '&' is to create the necessary space between equations.
du &= dx & dv &= \cos x
\end{align*}

\end{document}
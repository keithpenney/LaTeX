\documentclass{article}

\begin{document}

The square root ``sqrt" command takes one mandatory argument and one optional argument.  Without the optional argument, it prints like a square root. 

$\sqrt{2}$

With the optional argument set to $n$, it displays as the $n^{th}$ root.

$\sqrt[3]{6}$

Fractions can be displayed in multiple ways.  First, by simply typing 2/3, e.g. $2/3$.  Secondly, by using the ``frac" command, e.g. $\frac{2}{3}$ which has two required arguments.

More interestingly:

$$\frac{ab + ac}{2abc}$$

Contrast with the same code ``inline": $\frac{ab + ac}{2abc}$

We can use the ``displaystyle" command to force the display size on the inline math: e.g. $\displaystyle{\frac{ab + ac}{2abc}}$


How about some Greek characters?  Just type the name of the character as the command name, e.g. $\alpha$, $\beta$, $\gamma$, $\delta$, etc.  We can print capital Greek characters by capitlalizing their name in the command.  Watch out though!  If the capital Greek letter is identical to the Roman character on your keyboard (e.g. capital $\alpha$ is just A), the capitalized command does not exist and will cause an error in LaTeX.  e.g. $\Gamma$, $\Delta$, $\Theta$, etc.


Some other special characters:

$\infty$
$\nabla$
$\flat$
$\ell$
$\imath$
$\hbar$
$\sum$
$\prod$
 

Let's do some trig.  Using only symbols: $\cos 2\pi \theta$.  Using arguments in the ``cos" command: $\cos{2\pi\theta}$.  They print identically.  Here's the definition of the tangent function:

$$\tan \phi = \frac{\sin \phi}{\cos \phi}$$

Logarithmic functions work in the same way.  The base of the log (optional) is just an underscored character.

$$\ln e^x = x$$
$$\log_{10} 10^x = x$$

\end{document}
